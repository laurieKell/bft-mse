\documentclass[a4paper,10pt]{article}
%\documentclass[a4paper,10pt]{scrartcl}

\usepackage[utf8]{inputenc}
\usepackage[authoryear]{natbib}

\begin{document}

\section{Scenarios}

Table x (PossibleModellingApproachesNoFigs131205.xlsx) summarises the different sources of uncertainty, i.e
process, observation, estimation, implementation, model, value, translational or institutional. In this study
only model and value uncertainty are considered and possible modelling approaches are summarised in the final
column. These are summarised and factors and levels in table \ref{tab:scn}. The $3_{rd}$ column summarises the 
number of levels for each factor. The $4_{th}$ column shows the cummulative number of scenarios if only the 
main effects are modelled, i.e. one a single level is varied in the base case at a time. The $5_{th}$ column 
shows the cummulative number of scenarios if all interactions between factors are considered.

\textit{[The second block are scenarios that we may discuss but not actually run]}


The first two factors \textbf{Historic Catch} and \textbf{Future Recruitment} were the sources of uncertainty 
included in the assessment and projections used to calculate the K2SM.
The values of steepness chosen were 1 (as assumed in the assessment) and 0.7 an arbitary value to provide some 
contrast. Natural mortality was either that assumed by the working group (SCRS) or derived from weight-at-age
(Lorenzen, 1996). To evaluate the effect of artisanal fisheries juvenile mortality was increased by a factor of 
(0, 0.5 or 1). Plus group dynamics were evaluated for an increase in mortality and by setting the $F_{ratio}$ to 1.
The working group had estimated the $F_{ratio}$.

Mis-specification in growth parameters were evaluated for k and $L_{\infty}$. This was done by generating a 
length frequency from the numbers, length and CV of length-at-age for each year and then using slicing
slicing to generate a new catch-at-age matrix.

Two populations were modelled which had the same populations parameters but with different historic catch ratios, i.e.
1:1 or 1:2.

For bluefin in common with many stocks stock biomass, reference points,
stock recruitment relationships and projections are based on spawning stock biomass (SSB) and spawner per recruit (SPR); which assumes that fecundity is 
proportional to mass-at-age irrespective of the demographic composition of adults \cite{murawski2001impacts} and that somatic growth is
time-invariant. However, egg production to those based on SSB. We then evaluate the sensitvity of reference points and stock recruitment relationships 
to the level of biological knowledge incorporated into assessment advice and discuss the consequences for management and the use of 
meta analysis to develop priors for key parameters used in stock assessment. However, the growing literature describing maternal effects on different aspects of
population dynamics including fecundity, viability of eggs and larvae, the temporal and spatial structure of spawning and possibly recruitment for these 
reasons Marshall (2009) argues that rather than SSB total egg production should be used.



%fc=c(2,3,2,2,2,2,2,3,3,2,2)
% 
%cumsum(fc-c(0,rep(1,length(fc)-1)))
%cumprod(fc)
  
   
\begin{table}
\begin{center}
\label{tab:datasumm}
\begin{tabular}{|ccccc|}
\hline
{\tiny Factor}                	& {\tiny Levels} 				& {\tiny N}	& {\tiny $\Sigma$ Main Effects}	& {\tiny $\Sigma$ Interactions}\\				
\hline\hline
{\tiny Historic Catch} 		& {\tiny Reported, Inflated}			& {\tiny 2}	& {\tiny 2}			& {\tiny 2}\\				
{\tiny Future Recruitment}	& {\tiny medium,low,high}			& {\tiny 3}	& {\tiny 4}			& {\tiny 6}\\ 	
{\tiny Steepness}		& {\tiny 1, 0.7}				& {\tiny 2}  	& {\tiny 5}			& {\tiny 12}\\
{\tiny Natural Mortality}	& {\tiny SCRS, Life History}			& {\tiny 2}  	& {\tiny 6}			& {\tiny 24}\\
{\tiny Juvenile Mortality} 	& {\tiny $M_1 \times$ (1,1.5)}   		& {\tiny 2}  	& {\tiny 7}			& {\tiny 48}\\
{\tiny Plus Group Mortality} 	& {\tiny $M_{PG} \times$ (1, 2)}		& {\tiny 2}  	& {\tiny 8}			& {\tiny 96}\\
{\tiny Plus Group Fratio} 	& {\tiny SCRS, 1.0}				& {\tiny 2}  	& {\tiny 9}			& {\tiny 192}\\
\hline\hline
{\tiny ALK \& k} 		& {\tiny k $\times$ (1, 0.75,1.25)}		& {\tiny 3}  	& {\tiny 11}			& {\tiny 576}\\
{\tiny ALK \& $L_{\infty}$} 	& {\tiny  $L_{\infty}$ $\times$ (1,0.75,1.25)}	& {\tiny 3}  	& {\tiny 12}			& {\tiny 1728}\\
{\tiny 2 Populations} 		& {\tiny 1:1,1:2}				& {\tiny 2}  	& {\tiny 14}			& {\tiny 3456}\\
{\tiny SRP} 			& {\tiny SSB, TEP}				& {\tiny 2}	& {\tiny 15}			& {\tiny 6912}\\
\hline
\end{tabular}
\end{center}
\caption{Scenarios}
\label{tab:scn}
\end{table}

\clearpage
\section{References}
Lorenzen, K. "The relationship between body weight and natural mortality in juvenile and adult fish: a comparison of natural ecosystems 
and aquaculture." Journal of fish biology 49.4 (1996): 627-642.\\


Marshall, T., (2009). Implementing information on stock reproductive potential in fisheries management: The motivation, challenges and opportunities.
In Jakobsen, T., Fogarty, M.J., Megrey, B. A., Monksness, E., (Eds), Fish reproductive biology. Implications for assessment and management. Wiley-BlackWell. A john Wiley
\& Sons, Ltd., Publication. Oxford, UK. pp. 395-420.

\end{document}
