\documentclass[a4paper,10pt]{article}
%\documentclass[a4paper,10pt]{scrartcl}

\usepackage[utf8]{inputenc}
\usepackage[authoryear]{natbib}

\title{}

\begin{document}
\maketitle

\section*{Introduction}

Although several sources of uncertainty were considered when formulating the East Atlantic and 
Mediterranean Bluefin Tuna Recovery Plan, not all sources of uncertainty were explicitly considered. 
Therefore, a contract for a Risk Assessment was awarded under Phase III to identify the main 
sources of uncertainty and concerns of a wide range of stakeholders. 
Subsequently the meeting on bluefin stock assessment methods (SCRS 2013) endorsed this work and
reccommended that the the major sensitivities for both separate and mixed stock assessments (e.g., M, fecundity
schedule, SRR and alternative mechanism of population regulation) should be identified and that
a paper on a Risk assessment be written to inform OM scenarios based on the qualitative identification 
of uncertainty. Therefore in this study we turn this 
initial qualitative study into a quantitative one that can be used to weight the 
plausibility of simulation trials (i.e., scenarios) for the Operating Model (OM) used 
in the Management Strategy Evaluation (MSE).

When building an OM it is necessary to develop hypotheses about system dynamics that can be
run as part of stochastic Monte Carlo simulations. However, Monte Carlo simulations are costly 
in terms of time and resource to conduct. Therefore there is benefits in first running deterministic 
(or a limited number of stochastic) of simulations to identify main effects or important interactions. 
Following this fully stochastic simulations can be run for the trials (i.e. scenarios) that
are condidered to be important. 

Deterministic runs are conducted initially to explore the dynamics and the effect of model and value 
uncertainty, then stochastic simulations that include observation and process error and assessment 
procedures with feedback. Such an approach may also mean that rather than running all possible combinations 
of treatments, an experimental design can be used to run only main effects and selected interactions.


Process,Observation,Estimation,Implementation,Model,value,Translationa,Institutional

\section*{Uncertainty}

Scientific stock assessment advice given by the SCRS is presented in the form of the Kobe II Strategy Matrix (K2SM). 
Traditionally the K2SM shows the probabilities by year for different catches of achieving the management objective of ensuring 
that the stock biomass is greater than $B_{MSY}$ and fishing mortality less than $F_{MSY}$. 

\subsection*{PLE-116/2013}

ICCAT Resolution \textit{To Complete The Standardization Of The Presentation Of Scientific Information In The Scrs Annual Report [Res 13-16]
noted that the standardisation of information in the SCRS reports} had improved. However, it was thought important to distinguish 
between the inherent variability in the characteristics of the natural system that we are trying to mangage and uncertainty related 
to our knowledge of that system and our ability to monitor and control it. Therefore Res 13-16 resolved that 


 \begin{enumerate}
  \item The SCRS should clearly identify sources of variability and uncertainty and clearly explain how these variability and uncertainty 
        affect the stock assessment results and the interpretation of the Kobe II Strategy Matrices. 
  \item The SCRS should further enhance the standardization of information included in its reports.
  \item In addition to the minimum elements required by Resolution 11-14, the SCRS should further score the quality of the fishery 
        data and related to the knowledge of the system (e.g. biological parameters, fishery patterns history, selectivity) used as 
        inputs to stock assessments. Quality scores on data should be detailed and should summarize the state of knowledge of the 
        different inputs and report on:
  \begin{enumerate}[(a)]
    \item the quality, the reliability and, where relevant, the representativeness of input data and information, such as, but 
          not limited to, 
      \begin{enumerate}[(i)]
      \item fisheries statistics and fisheries indicators (e.g. catch and effort, catch-at-size and catch-at-age matrices 
          by sex and, when applicable, fisheries dependent indices of abundance), 
      \item biological information (e.g. growth parameters, 
          natural mortality, maturity and fecundity, migrations patterns and stock structure, fisheries independent indices of abundance) 
          and 
      \item complementary information (i.e. consistencies among available abundance indices, influence of the environmental factors 
          on the dynamic of the stock, changes in fishing effort distribution, selectivity and fishing power, changes in target species),
    \end{enumerate}
    \item limitations of the assessment models used with respect to the type and the quality of the input data,
    \item potential biases in the assessment results associated with uncertainties of the input data.
  \end{enumerate}
  \item For the purpose of paragraphs 3 and 4, the SCRS should consider a specific table or any other alternative format 
        to be included in its annual report in association to the Kobe plot in order to summarize the information required 
        in this resolution.
        The SCRS should clearly identify sources variability and uncertainties associated with the input data and available 
        information and specify on how these variability and uncertainties can affect the stock assessment results and the  
        interpretation of the information provided in the Kobe II Strategy Matrices. 
  \item In cases where the SCRS utilizes different modeling approaches and/or scenarios (i.a., sensitivity runs or 
        alternate hypotheses) to incorporate uncertainty in stock assessments, the SCRS should clearly identify the most 
        defensible approach or the most likely scenario (i.e., 'base case') and provide the rationale for its decision.
        In cases where these different approaches and/or scenarios would finally be considered as equivalent, the uncertainty 
        between the models should be accounted for in the calculation of the stock assessment parameters. 
\end{enumerate}
 

\subsection*{Projections, Forecasts and Scenarios}

While ICCAT uses the term projection for any procedure used to develop a K2SM in other fora the term projection has a
very specific meaning. For example the Intergovernmental panel on climate change (IPCC) differentiates between the terms projection,
forcast/prediction and scenario (IPCC, 1994). 

Projection refers to any model-derived estimates of the future; forecast (or prediction) is used when a projection is branded "most likely". 
While a scenario is a coherent, internally consistent, and plausible description of a possible future state of 
the world. A scenario is not a forecast but a single alternative image of how the future can unfold. 
A projection may serve as he raw material for a scenario and a set of scenarios is often adopted to reflect, as well as possible, 
the range of uncertainty in projections. It has been argued that if probabilities can be assigned 
to a range of predictions (while acknowledging that significant unquantifiable uncertainties outside the range remain), 
a new descriptor is required intermediate between scenario and forecast (Jones, 2000).

In fisheries the accepted way to model uncertainty consistent with the Precauntionary Approach is to use MSE

\subsection*{MSE}
Traditional stock assessments mainly considers only uncertainty in observations and process (e.g. recruitment). 
However, uncertainty about the actual dynamics (i.e. model uncertainty) has a larger impact on achieving management objectives (Punt 2008). 
Therefore when providing management advice it is important to consider appropriate sources of uncertainty.  Rosenberg and Restrepo (1994) catagorised uncertainties in fish stock 
assessment and management as:

\begin{itemize}
 \item Process error; caused by disregarding variability, temporal and spatial, in dynamic population and fisheries processes;
 \item Observation error;  sampling error and measurement error;
 \item Estimation error; arising when estimating parameters of the models used in the assessment procedure;
 \item Model error; related to the ability of the model structure to capture the core of the system dynamics;
 \item Implementation error;  where the effects of management actions may differ from those intended.
\end{itemize}

Sources of uncertainty related to \emph{Model Error} include
    
\begin{itemize}
 \item structural uncertainty; due to inadequate models, incomplete or competing conceptual frameworks, or where significant processes or 
relationships are wrongly specified or not considered. Such situations tend to be underestimated by experts (Morganand Henrion, 1990). and
 \item value uncertainty; due to missing or inaccurate data or poorly known parameters.
\end{itemize}

As part of the implementation of a management plan uncertainty related to how results are used also need to be considered (Francis and Shotton, 1997) and 
include 

\begin{itemize}
 \item translational uncertainty; in explaining uncertain results and 
 \item institutional uncertainty; due to lack of social capital (i.e. ways to handle these types of problems) in stakeholder organisations 
to cope with management issues. 
\end{itemize}

\subsection*{Factorial Design}

A full MSE requires appropriate uncertainty to be considered, particularly in the choice of OM scenarios. 
 
One way of doing this is base the choice of scenarios are based on a factorial design.  A full factorial experiment is one whose 
design consists of two or more factors, each with discrete possible values or \textit{levels}, and where experimental 
units take on all possible combinations of these levels across all factors. Such a design is better able to
represents the complexity of the real world and allows an evaluation of whether the effect of one factor
depends on the level of another factor. The potentially large number of combinations in a full factorial design 
may mean that it is not possible to run them all in the time available in a stock assessment working group.
Therefore a fractional factorial design in which some of the possible combinations are omitted may be prefered.

However, when conducting a Management Strategy Evaluation a large number of scenarios need to be considered to
evaluate the main sources of uncertainties. I.e. the Operating Models (OM) will need to be 
conditioned on a wider range of data and knowledge that routinely considered within a stock assessment. 

In other words while only a few scenarios are routinely be considered within an ICCAT stock assessment, many more 
scenarios will need to be run as part of an MSE. This presents a potential problem if
the Scenarios from an MSE result in different conclusions from those ran in a stock assessment.

It is proposed that a base case be proposed and then factors with levels that represent the main uncertainties. In the
stock assessment WG the main effects can be evaluated by varying 1 factor at a time. Hopefully this will 
allow the stock assessment to \textit{bracket} the main uncertainty and act as a simple screening experiment, 
to determine the factors have the greatest influence on the perception of stock dynamics.
Based on the identification of the most important factors, a multi-level designed experiment can then be developed
for the MSE that includes interactions between factors.

\newpage\clearpage
\bibliography{/home/laurie/Desktop/u1/papers/tex/include/refs.bib} 
\bibliographystyle{abbrvnat} 

\end{document}

