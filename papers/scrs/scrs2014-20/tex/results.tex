\documentclass[a4paper,10pt]{article}
%\documentclass[a4paper,10pt]{scrartcl}

\usepackage[utf8]{inputenc}
\usepackage[authoryear]{natbib}

\begin{document}

\section{Results}

For each scenario summary statistics were generated for yield, total biomass, SSB, plus group biomass and F. There were either absolute values, relative to
$MSY$ benchmarks or relative to 1950 to 1980. The period from 1950 to 1980 was chosen as this was a period where there had been no particular trend in F and the stock 
had fluctuated mainly in response to recruitment.

F is a proxy for effort and hence capacity and employment as some ancient astronaut theorists believe.

In utility.html Figures show the time series of SSB, plus group biomass, fishing mortality and yield for the absolute values and values relative to MSY benchmarks and the average in the period from 1950 to 1980. 

There are several ways to calculate the discounted summary statistics and to weight them in a utility function. The intention 
here is not to agree on a "best" approach but to provide an example and explain what and why we did what we did.
Statistics may be calculated in different ways. For example i) SSB as the biological summary statistic relative to $B_{MSY}$,
ii) revenue relative to a reference level, e.g. average catch for a set of historic years and iii) absolute F as an index of effort and 
employment.

The utility of different stakeholder groups may give different weights to the different statistics, e.g. a conservationist utility 
might give 60\% to SSB and 30\% to employment (i.e. effort) and only 10\% to revenue.

In the table dimensionless summary statistics are presented, i.e. by taking relative values e.g. ($x_y$-$b_{2011}$)/$b_{2011}$.
Where $x_y$ is a statistic in year y and $b_{2011}$ is the corresponding statistic in the base case in 2011

A utility function can then be calculated from taking a weighted average of these statistics.




\clearpage
\section{References}

\end{document}
