\documentclass{beamer}
\usepackage{beamerthemesplit}
\newcommand{\ds}{\displaystyle}
\newcommand{\eps}{\epsilon}
\newcommand{\veps}{\varepsilon}
\newcommand{\wh}{\widehat}
\usetheme{Berkeley}
\usecolortheme{dolphin}
\title{The CCSBT MP process (nice idea but how do you get it adopted and implemented?)}
\author{Too many to mention (but some are in the room): the member scientists and indepedent panel of the CCSBT SC}
\begin{document}
\maketitle
% frame 1
\begin{frame}
\frametitle{Talk outline}
\begin{itemize}
    \item Getting OM structure defined \& agreed
    \item Uncertainty:
        \begin{enumerate}
            \item[(i)] \textbf{Quantitative}: things we believe we can either estimate or define \textit{a priori}
            \item[(ii)] \textbf{Qualitative}: scenarios we think are likely, possibly even ranked, but not able to probabilistically decide between
        \end{enumerate}
    \item What data do we want to be part of an MP?
    \item Targets, operational constraints and performance measures
    \item From many to two: iterative deselection and adaption
    \item Two become one: what to do when it's close \emph{and} tense...
    \item Actual adopted MP specs and performance
    \item Lessons learned and future MP and OM development
\end{itemize}
\end{frame}
% frame 2
\begin{frame}
\frametitle{CCSBT Operating Model}
\begin{itemize}
\item Previously more than one assessment model and different data interpretations
\item Part of previous MSE work (prior to over-catch) single OM structure developed
\item Important to set the base for the MSE here - already too many options later on; get this agreed first...
\end{itemize}
\end{frame}
% frame 3
%\begin{frame}
%\frametitle{Quick-capsule life-history}
%\begin{itemize}
%\item CCSBT areas: \textbf{juveniles} (summer, area 3 (GAB); winter areas 4-8), \textbf{sub-adults/adults} (feeding, area 9), \textbf{spawners} (staging ground, area 2; spawning ground, area 1 \textit{ca.} Oct-Apr)
%\begin{figure}
%\begin{center}
%\hspace{-1cm}\includegraphics[width=8cm,height=4cm]{CCSBT_areas.pdf}
%\end{center}
%\end{figure}
%\end{itemize}
%\end{frame}
% frame 4
\begin{frame}
\frametitle{OM structure}
\begin{itemize}
\item Seasonal, non-spatial age/length structured model
\item Data:
    \begin{enumerate}
        \item Catch biomass and composition (age/length)
        \item CPUE (Japanese long-line fleet from 1969 onwards)
        \item Mark-recapture data (from large-scale tagging in 1990s)
        \item Fishery independent aerial survey of juvenile SBT
    \end{enumerate}
\item Key parameters estimated directly:
    \begin{enumerate}
        \item Unfished SSB
        \item Recruitment deviates
        \item Natural mortality (ages 4 and 30+, specified func. form)
        \item Selectivity (time-varying) for each fishery
    \end{enumerate}
\end{itemize}
\end{frame}
% frame 5
\begin{frame}
\frametitle{Quantitative uncertainties: the ``grid''}
\begin{itemize}
    \item Key parameters/options where may not have convincing information in data \emph{but} can explore quantitative options:
        \begin{enumerate}
            \item {\color{orange} Steepness (h)}: strong but one-way trip decline in CPUE...
            \item {\color{orange}$M_{0,10}$}: no direct data but vital to define shape of $M_a$
            \item {\color{orange} $\omega$}: non-linearity of biomass-to-CPUE relationship
            \item {\color{orange} CPUE series}: spatial weighting options for core series
            \item {\color{orange} $q$ age-range}: ages over which LL CPUE $q$ calculated
            \item {\color{orange} Sample size}: initial effective sample sizes 
        \end{enumerate}
    \item With grid elements have pre-defined priors \emph{but} option of resampling based on objective function
\end{itemize}
\end{frame}
% frame 6
\begin{frame}
\frametitle{Quantitative uncertainties: the ``grid''}
\begin{itemize}
    \item For MSE work grid option table (reference set of OMs):
\begin{table}
\begin{center}
\label{tab:datasumm}
\begin{tabular}{|cccccc|}
\hline
& {\tiny Levels} & {\tiny CumulN} & {\tiny Values} & {\tiny Prior} & {\tiny Weighting}\\
\hline\hline
{\tiny $h$} & {\tiny 5} & {\tiny 5} & {\tiny 0.55, 0.64, 0.93, 0.82, 0.9} & {\tiny uniform} & {\tiny obj. fun.}\\
{\tiny $M_0$} & {\tiny 4} & {\tiny 20} & {\tiny 0.3, 0.35, 0.4, 0.45} & {\tiny uniform} & {\tiny obj. fun.}\\
{\tiny $M_{10}$} & {\tiny 3} & {\tiny 60} & {\tiny 0.07, 0.1, 0.14} & {\tiny uniform} & {\tiny obj. fun.}\\
{\tiny $\omega$} & {\tiny 1} & {\tiny 60} & {\tiny 1} & {\tiny NA} & {\tiny NA}\\
{\tiny CPUE} & {\tiny 2} & {\tiny 120} & {\tiny w.5, w.8} & {\tiny uniform} & {\tiny prior}\\
{\tiny $q$ age-range} & {\tiny 2} & {\tiny 240} & {\tiny 4-18, 8-12} & {\tiny 0.67, 0.33} & {\tiny prior}\\
{\tiny Sample size} & {\tiny 1} & {\tiny 240} & {\tiny SQRT} & {\tiny NA} & {\tiny NA}\\
\hline
\end{tabular}
\end{center}
\end{table}
\item From 240 grid permutations sample of 2000 generated
\end{itemize}
\end{frame}
% frame 7
\begin{frame}
\frametitle{Reference set OM SSB \& recruitment}
\begin{itemize}
    \item SSB (top, 1e+6t) and recruitment (bottom, millions) median \& 80\% CI
\begin{figure}
\begin{center}
\vspace{-0.75cm}
%\hspace{-1cm}\includegraphics[width=8cm,height=7cm]{stkrec.pdf}
\end{center}
\end{figure}
\end{itemize}
\end{frame}
% frame 8
\begin{frame}
\frametitle{Qualitative uncertainties: robustness trials}
\begin{itemize}
    \item Location for alternative OM settings/future scenarios
    \item For SBT main issues:
        \begin{enumerate}
            \item Over-catch scenarios (magnitude, CPUE impact)
            \item CPUE and survey variability
            \item Tag mixing
            \item Non-linearity in biomass-to-CPUE
            \item Alternative CPUE series (model/data)
            \item Catchability changes (past/future) given LL changes
        \end{enumerate}
    \item Core set of plausible robustness trials chosen
\end{itemize}
\end{frame}
% frame 9
\begin{frame}
\frametitle{Uncertainty overall}
\begin{itemize}
    \item Recruitment stochasticity ($\sigma_R=0.6$) - projections
    \item Observation/process error in CPUE/survey - projections
    \item Catch composition - projections
    \item No direct inclusion of parametric uncertainty (grid only)
    \item Theory (hope) is robustness tests cover plausible range
    \item No implementation error assumed in projections
\end{itemize}
\end{frame}
% frame 10
\begin{frame}
\frametitle{What data can candidate MPs use?}
\begin{itemize}
    \item Previous catch, CPUE, aerial survey, LL age composition
    \item Have to use \textit{at least} CPUE \textbf{and} aerial survey
    \item Median and 80\%CI for aerial survey (top) and LL CPUE (bottom): 
\begin{figure}
\begin{center}
\hspace{-1cm}
%\includegraphics[width=7cm,height=5cm]{data_plot.pdf}
\end{center}
\end{figure}
\end{itemize}
\end{frame}
% frame 11
\begin{frame}
\frametitle{Targets \& Operational Constraints}
\begin{itemize}
    \item SSB rebuilding target: $p(SSB_{y^{\rm targ}}>0.2 SSB_0)=p^{\rm targ}$ 
    \item Target year: $y^{\rm targ}$ either 2035 or 2040
    \item Target probability: $p^{\rm targ}$ 0.6, 0.7 and 0.9
    \item Control variable: global TAC (for fixed national allocation)
    \item Frequency: every 2 or 3 years 
    \item Implementation lag: zero or one year
    \item Minimum TAC change: 100t 
    \item Maximum TAC change: 3000t/5000t
\end{itemize}
\end{frame}
% frame 12
\begin{frame}
\frametitle{Midpoint observation}
\begin{itemize}
    \item Lots of switches and options $\equiv$ \textbf{unhelpful}
    \item If possible quickly explore what matters, what doesn't
    \item Target level, year \& probability alias 
    \item Virtually no difference between 2 and 3 years (long-lived)
    \item Give clear and early advice to Commissioners:
        \begin{enumerate}
            \item Avoid specifics until OM and robustness trials defined
            \item Plan for and expect multiple iterations
            \item Remove quasi-identitical scenarios 
        \end{enumerate}
    \item Makes understanding what matters easier \& faster
    \item Increases likelihood of acceptance later on...
\end{itemize}
\end{frame}
% frame 13
\begin{frame}
\frametitle{Performance measures}
\begin{itemize}
    \item Focus of (interim) MP: rebuild SSB to 20\% $SSB_0$
    \item Base criteria: MP ``tuned'' to targets on reference set OMs
    \item Tuned MP then run on all relevant robustness trials
    \item Key SSB performance measures:
        \begin{enumerate}
            \item $p(SSB_{\rm fut}<SSB_{\rm min})$: future declines
            \item $p(SSB_{2022/2025}>0.2 SSB_0)$: half-way point
            \item $\mathbb{E}(SSB_{\rm fut}/SSB_{2011})$: rebuilding factor
        \end{enumerate}
    \item Key catch performance measures:
        \begin{enumerate}
            \item Average future catch (range of years)
            \item Average annual variation (AAV)
            \item Max. TAC decrease
            \item $p(C_{t}\uparrow\,|\,C_{t-\tau}\downarrow)$: down-then-up catch trajectories
        \end{enumerate}
       \item Create 47,000 figures \& tables...
\end{itemize}
\end{frame}
% frame 14
\begin{frame}
    \frametitle{Design \& initial selection process}
\begin{itemize}
    \item Member scientists invited to design candidate MPs
    \item Base criteria: hit targets on reference set of OMs
    \item Range of candidate MPs:
        \begin{enumerate}
            \item Purely empirical with different HCRs
            \item Model-based (production, relative abundance)
            \item Fuzzy logic even made an appearance
        \end{enumerate}
    \item Core group of CCSBT SC met at technical workshop
\end{itemize}
\end{frame}
% frame 15
\begin{frame}
    \frametitle{From many to two}
\begin{itemize}
    \item Workshop chose two MPs to go to CCSBT SC in 2010
    \item Both clearly performed best of suite of candidates
    \item $MP_1$:
        \begin{enumerate}
            \item Model-based (recruit-adult relative abundance model)
            \item CPUE target-driven
            \item Recruitment limit-type behaviour
        \end{enumerate}
    \item $MP_2$:
        \begin{enumerate}
            \item Empirical (using CPUE and aerial survey)
            \item CPUE trend-driven
            \item Recruitment target-type behaviour
        \end{enumerate} 
\end{itemize}
\end{frame}
% frame 16
\begin{frame}
    \frametitle{CCSBT 2010}
\begin{itemize}
    \item Performance of $MP_1$ and $MP_2$ scrutinised in detail
    \item $MP_1$ summary:
        \begin{enumerate}
            \item lower initial TACs; higher average TACs 
            \item less catch and SSB variability
            \item better on pessimistic robustness trials
        \end{enumerate}
    \item $MP_2$ summary:
        \begin{enumerate}
            \item higher initial TACs; lower average TACs
            \item more variability in catch and SSB
            \item better on catchability change trials
        \end{enumerate}
    \item Hard to choose...
    \item Individual and average MP go to Commission
    \item Commission can't decide, wants more options and work...
\end{itemize}
\end{frame}
% frame 17
\begin{frame}
    \frametitle{CCSBT 2011}
\begin{itemize}
    \item Minor tweaks but $MP_1$ and $MP_2$ back again...
    \item CCSBT OM reconditioned with latest data (more positive)
    \item Target probability agreed: 0.7; TACs every 3 years
    \item Still in play: 2035 or 2040; 3000t or 5000t; lag or no lag
    \item \textbf{Issue No. 1}: one MP can't tune to likely settings
    \item \textbf{Issue No. 2}: issues around ``ownership'' of MP
    \item Tuning issue means average off the table...
\end{itemize}
\end{frame}
% frame 18
\begin{frame}
    \frametitle{Two become one (fast)}
\begin{itemize}
    \item Looks like no agreement probable with \textit{status quo}
    \item 3 days before end of meeting ``fusion'' MP coded
    \item Introduced to SC for discussion
    \item SC: if it tunes on problem scenarios we go for it (it did)
    \item Every available laptop confiscated
    \item Fusion MP (originally $MP_3$) tuned, robustness trials run
    \item Good performance balance relative to original MPs
    \item SC recommends $MP_3$ to Commission for consideration
\end{itemize}
\end{frame}
% frame 19
\begin{frame}
    \frametitle{2011 Commission outcomes}
\begin{itemize}
    \item The ``Bali Procedure'' adopted by CCSBT Commission:
        \begin{enumerate}
            \item Target year: 2035
            \item Maximum TAC change: 3000t
            \item 1 year implementation lag (except for 2012)
         \end{enumerate}
     \item Initial TAC increase limited to 1000t (new TAC 10,449t)
     \item TAC locked for 2012/2013 
     \item 2014 TAC min. of 12,449t or level from MP (2015-2017)
\end{itemize}
\end{frame}
% frame 20
\begin{frame}
    \frametitle{Specifications of CCSBT MP}
\begin{itemize}
    \item Key data: LL CPUE \& aerial survey
    \item RE model for juvenile ($R_y$), adult ($B_y$) biomass:
        \begin{align*}
            \ds B_{y} &= R_{y}+g_{y-1}B_{y-1},\\ 
            \ds R_y &= \exp\left(\mu_R+\eps^R_y\right),\\
            \ds g_y &= \exp\left(\mu_g+\eps^g_y\right),\\
            \ds \eps^\bullet_y &\sim N\left(-\sigma^2_\bullet/2,\sigma^2_\bullet\right).
        \end{align*}
    \item Aerial survey: $I^{AS}_{y-1}\sim R_y$
    \item LL CPUE: $I^{CPUE}_y\sim B_y$
\end{itemize}
\end{frame}
% frame 21
\begin{frame}
    \frametitle{Specifications of CCSBT MP}
\begin{itemize}
    \item Mini-assessment central to MP:
\begin{figure}
\begin{center}
\hspace{-1cm}
%\includegraphics[width=6cm,height=6cm]{fits_pvals.pdf}
\end{center}
\end{figure}
\end{itemize}
\end{frame}
% frame 22
\begin{frame}
    \frametitle{Specifications of CCSBT MP}
\begin{itemize}
    \item HCR is an average of two candidate TACs:
        \begin{equation*}
            \ds TAC^1_{y+1}=TAC_y\times \left\{\begin{array}{rcl}\ds{1-k_1|\lambda|^{\gamma}} & \mbox{for} & \lambda<0\\[0.35cm]
\ds{1+k_2\lambda} & \mbox{for} & \lambda\geq 0 
    \end{array}\right.
        \end{equation*}
    \item Above $\lambda$ is the slope in the regression of $\ln B_y$ against year (from years $y-\tau_B+1$ to year $y$)
    \item $k_1$ and $k_2$ ``gain'' parameters; $\gamma\geq1$ action asymmetry
\end{itemize}
\end{frame}
% frame 23
\begin{frame}
    \frametitle{Specifications of CCSBT MP}
\begin{itemize}
    \item Second candidate TAC:
        \begin{align*}
            \ds TAC^2_{y+1} &= 0.5\times\left(TAC_y+C^{\rm targ}_y\Delta^R_y\right),\\
                \ds C^{\rm targ}_y &= \left\{\begin{array}{rcl}\ds{\delta \left[\frac{B_{y}}{B^*}\right]^{1-\veps_b}} & \mbox{for} & B_{y}\geq B^*\\[0.35cm]
\ds{\delta \left[\frac{B_{y}}{B^*}\right]^{1+\veps_b}} & \mbox{for} & B_{y}<B^*
    \end{array}\right.,\\
\ds \Delta^R_y &= \left\{\begin{array}{rcl}\ds{\left[\frac{\bar{R}}{\mathcal{R}}\right]^{1-\veps_r}} & \mbox{for} & \bar{R}\geq\mathcal{R}\\[0.35cm]
\ds{\left[\frac{\bar{R}}{\mathcal{R}}\right]^{1+\veps_r}} & \mbox{for} & \bar{R}<\mathcal{R}
\end{array}\right.
        \end{align*}
\end{itemize}
\end{frame}
% frame 24
\begin{frame}
    \frametitle{Specifications of CCSBT MP}
\begin{itemize}
    \item $\delta$ ``target'' catch (tuning parameter)
    \item $B^*$: ``target'' CPUE (mean CPUE obs. @ 20\% $SSB_0$)
    \item $\bar{R}$ average recent juvenile biomass:
      \begin{equation*}
      \ds \bar{R}=\frac{1}{\tau_R}\sum\limits_{i=y-\tau_R+1}^{y}R_i,
      \end{equation*}
    \item $\mathcal{R}$: ``limit'' level; mean survey 1993-2011
    \item $\veps^\bullet\in[0,1]$ action asymmetry 
\end{itemize}
\end{frame}
% frame 25
\begin{frame}
    \frametitle{Specifications of CCSBT MP}
\begin{itemize}
    \item Overall: $TAC_{y+1}=0.5\times\left(TAC^1_{y+1}+TAC^2_{y+1}\right)$ 
    \item Control parameters:
\begin{table}[ht]
\begin{center}
\begin{tabular}{|cc|}
\hline
Parameter & BP\\
\hline\hline
$\delta$ & 30,750\\
$k_1$ & 1.5\\
$k_2$ & 3\\
$\gamma$ & 1\\
$\tau_B$ & 7\\
$B^*$ & 1.2\\
$\veps_b$ & 0.25\\
$\veps_r$ & 0.75\\
$\tau_r$ & 5\\
\hline
\end{tabular}
\end{center}
\end{table}
\end{itemize}
\end{frame}
% frame 25a
\begin{frame}
    \frametitle{Bali Procedure performance}
\begin{itemize}
    \item Detailed summary for key robustness tests:
\begin{figure}
\begin{center}
%\vspace{-0.75cm}
\hspace{-1cm}
%\includegraphics[width=7cm,height=7cm]{Compare_Group_Stats_2035.pdf}
\end{center}
\end{figure}
\end{itemize}
\end{frame}
% frame 26
\begin{frame}
    \frametitle{Bali Procedure performance}
\begin{itemize}
    \item Performance summary:
\begin{table}[ht]
\begin{center}
\begin{tabular}{|ccccc|}
\hline
{\tiny Scenario} & {\tiny $p(SSB_{2035}>0.2 SSB_0)$} & {\tiny $\mathbb{E}\left(\frac{SSB_{2022}}{SSB_{2011}}\right)$} & {\tiny $\mathbb{E}(C_{2012-2022})$} & {\tiny $p(C\uparrow\downarrow)$}\\
\hline\hline
{\tiny Ref. 3000t} & {\tiny 0.7} & {\tiny 2.76} & {\tiny 15,200} & {\tiny 0.49}\\
{\tiny Ref. 5000t} & {\tiny 0.7} & {\tiny 2.65} & {\tiny 15,600} & {\tiny 0.71}\\
& {\tiny \textbf{Robustness trials }} & & & \\
{\tiny lowR, 3000t} & {\tiny 0.66} & {\tiny 2.32} & {\tiny 13,200} & {\tiny 0.83}\\
 {\tiny Upq, 3000t} & {\tiny 0.45} &  {\tiny 2.58} & {\tiny 15,300} & {\tiny 0.5}\\
{\tiny STWin, 3000t} & {\tiny 0.34} & {\tiny 2.39} & {\tiny 12,872} & {\tiny 0.81}\\
{\tiny Omega75, 3000t} & {\tiny 0.48} & {\tiny 2.74} & {\tiny 13,304} & {\tiny 0.74}\\
\hline
\end{tabular}
\end{center}
\end{table}
\end{itemize}
\end{frame}
% frame 27
\begin{frame}
    \frametitle{Bali Procedure performance}
\begin{itemize}
    \item ``Worm'' plot (reference OM):
\begin{figure}
\begin{center}
\vspace{-0.75cm}
\hspace{-1cm}
%\includegraphics[width=7cm,height=7cm]{worms.pdf}
\end{center}
\end{figure}
\end{itemize}
\end{frame}
% frame 28
\begin{frame}
    \frametitle{Lessons learned (technical)}
\begin{itemize}
    \item Tuning is \textbf{essential} - main point of reference set OMs
    \item Crucial comparability baseline between candidate MPs
    \item ``Entry level'' test: no tuning, no further...
    \item Use pessimistic robustness tests to rule out riskier MPs
    \item Detailed robustness performance for final ``decisions''
\end{itemize}
\end{frame}
% frame 28a
\begin{frame}
    \frametitle{Lessons learned (general)}
\begin{itemize}
    \item Going to take a number of years; plan accordingly
    \item Agree and codify ``exceptional circumstances''
    \item Clear guidance to Commissioners right from start
    \item Minimise options given to Commissioners
    \item Encourage full collaborative developement
    \item For CCSBT ``Olympic'' approach led to stalemate
    \item Get ``member ownership'' off the table early
\end{itemize}
\end{frame}
% frame 29
\begin{frame}
    \frametitle{Future OM \& MP work}
\begin{itemize}
    \item Develop spatial OMs (use tagging/e-tagging data)
    \item Include close-kin genetics into OM (adult abundance)
    \item Adopted MP is \emph{interim} (get to limit level)
    \item MP in place is good but not a recovery guarantee:
        \begin{enumerate}
            \item Keep exploring cost-effective monitoring
            \item Future development of adjusted/alternative MPs
            \item Time does not stand still post implementation...
        \end{enumerate}
\end{itemize}
\end{frame}
% frame 30
\begin{frame}
    \frametitle{Acknowledgements}
\begin{itemize}
    \item Over many years member scientists, Commissioners and indepedent panel of the CCSBT SC
\end{itemize}
\end{frame}

\end{document}
